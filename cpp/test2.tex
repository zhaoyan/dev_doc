\documentclass{article}
\usepackage{lipsum}

\usepackage{listings}
\usepackage{totcount}


\newcounter{maxlstnumber}
\regtotcounter{maxlstnumber}
\def\updatemaxlstnumber{%
  \ifnum\value{lstnumber}>\value{maxlstnumber}%
  \setcounter{maxlstnumber}{\the\value{lstnumber}}%
  \fi%
}
\lstset{numbers=left, gobble=2, numbersep=5pt, firstnumber=11109, escapeinside={//*}{*//}}
\newlength{\MaxSizeOfLineNumbers}%

\makeatletter
% The following command allows you to customize line number style, without affecting \ref{}.
% Here, the style is "\thelstnumber." (with a dot at the end)
\def\renderlstnumber{\normalfont\lst@numberstyle{\thelstnumber.}\kern\lst@numbersep}
\def\lst@PlaceNumber{\updatemaxlstnumber\makebox[\MaxSizeOfLineNumbers][r]{\renderlstnumber}}
\makeatother

\begin{document}
%\autosizelstnumber

\parindent 0pt

%\lipsum[1]

\begin{lstlisting}
  aline
  bline//*\label{bline}*//
  yet another line//*\label{largestline}*//
\end{lstlisting}

Line \ref{bline} contains the character \verb!b!. Notice that even
though the line numbering says ``\ref{bline}.'', the
\verb!\ref{bline}! used here says ``\ref{bline}''. The dot is taken
into account when calculating the width to reserve for line numbers.

\begin{lstlisting}[firstnumber=123]
  class Foo {
    public static void main() {}
  }//*\label{lastline}*//
\end{lstlisting}

Notice that even if we end with line \ref{lastline}, the with reserved
for line numbers is large enough to include \ref{largestline}, the
largest line.

\lipsum[2]

\end{document}